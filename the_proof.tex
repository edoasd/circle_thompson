\section{Strategy for the proof of Theorem \ref{thm: main thm circle Thompson}}

\begin{enumerate}
	\item Consider a finitely supported non-degenerate probability measure $\mu$ on $T$, and denote by $(S^1,\nu)$ the associated $\mu$-boundary given by the circle endowed with the corresponding stationary measure.
	\item We are going to prove the following statement.
	\begin{prop}
		For $\lambda$-almost every $\xi \in S^1$ we have $h^{\xi}(\mu)>0$.
	\end{prop}
	\item We just need to prove that for $\lambda$-almost every $\xi \in S^1$ there is $C>0$ such that for every $n$ sufficiently large, we have $H_{\xi}(w_n)\ge Cn$.
	\item Note that $h^{\xi}(\mu^{*N})=Nh^{\xi}(\mu)$. Hence, since $\mu$ is non-degenerate, we may assume that the support of $\mu$ contains:
	\begin{itemize}
		\item The identity element,
		\item a non-trivial element supported on $[1/4,3/4]$,
		\item a non-trivial element supported on $[1/2,1]$,
		\item a non-trivial element supported on $[3/4,1]\cup [0,1/4]$, and
		\item a non-trivial element supported in $[0,1/2]$.
	\end{itemize}
	\item Let us consider boundary points $\xi \in [3/8,5/8]$. Let $a$ be the identity element, and $b$ be the non-trivial element supported on $[3/4,1]\cup [0,1/4]$.
	\item Let us consider trajectories of the $\mu$-random walk for which $\xi$ is the boundary point of $S^1$.
\end{enumerate}